% !TEX encoding = cp1250
\documentclass[a4paper,titlepage,11pt,twosides,floatssmall]{mwrep}
\usepackage[left=2.5cm,right=2.5cm,top=2.5cm,bottom=2.5cm]{geometry}
\usepackage[OT1]{fontenc}
\usepackage{polski}
\usepackage{amsmath}
\usepackage{amsfonts}
\usepackage{amssymb}
\usepackage{graphicx}
\usepackage{float}
\usepackage{url}
\usepackage{tikz}
\usetikzlibrary{arrows,calc,decorations.markings,math,arrows.meta}
\usepackage{rotating}
\usepackage[percent]{overpic}
\usepackage[cp1250]{inputenc}
\usepackage{xcolor}
\usepackage{colortbl}
\usepackage{pgfplots}
\usetikzlibrary{pgfplots.groupplots}
\usepackage{listings}
\usepackage{matlab-prettifier}
\usepackage{enumitem,amssymb}
\definecolor{szary}{rgb}{0.95,0.95,0.95}
\usepackage{tabularx}
\usepackage{siunitx}
\usepackage{makecell}
\sisetup{detect-weight,exponent-product=\cdot,output-decimal-marker={,},per-mode=symbol,binary-units=true,range-phrase={-},range-units=single}
\SendSettingsToPgf
%konfiguracje pakietu listings
\lstset{
	backgroundcolor=\color{szary},
	frame=single,
	breaklines=true,
}
\lstdefinestyle{customlatex}{
	basicstyle=\footnotesize\ttfamily,
	%basicstyle=\small\ttfamily,
}
\lstdefinestyle{customc}{
	breaklines=true,
	frame=tb,
	language=C,
	xleftmargin=0pt,
	showstringspaces=false,
	basicstyle=\small\ttfamily,
	keywordstyle=\bfseries\color{green!40!black},
	commentstyle=\itshape\color{purple!40!black},
	identifierstyle=\color{blue},
	stringstyle=\color{orange},
}
\lstdefinestyle{custommatlab}{
	captionpos=t,
	breaklines=true,
	frame=tb,
	xleftmargin=0pt,
	language=matlab,
	showstringspaces=false,
	basicstyle=\footnotesize\ttfamily,
	%basicstyle=\scriptsize\ttfamily,
	keywordstyle=\bfseries\color{green!40!black},
	commentstyle=\itshape\color{purple!40!black},
	identifierstyle=\color{blue},
	stringstyle=\color{orange},
}

%wymiar tekstu (bez �ywej paginy)
\textwidth 160mm \textheight 247mm

%ustawienia pakietu pgfplots
\pgfplotsset{
tick label style={font=\scriptsize},
label style={font=\small},
legend style={font=\small},
title style={font=\small}
}

\def\figurename{Rys.}
\def\tablename{Tab.}

%konfiguracja liczby p�ywaj�cych element�w
\setcounter{topnumber}{0}%2
\setcounter{bottomnumber}{3}%1
\setcounter{totalnumber}{5}%3
\renewcommand{\textfraction}{0.01}%0.2
\renewcommand{\topfraction}{0.95}%0.7
\renewcommand{\bottomfraction}{0.95}%0.3
\renewcommand{\floatpagefraction}{0.35}%0.5

\begin{document}
\frenchspacing
\pagestyle{uheadings}

%strona tytu�owa
\title{\bf Sprawozdanie z �wiczenia nr 1\vskip 0.1cm}
\author{Tymon Kobylecki}
\date{2022}

\makeatletter
\renewcommand{\maketitle}{\begin{titlepage}
\begin{center}{\LARGE {\bf
Wydzia� Elektroniki i Technik Informacyjnych}}\\
\vspace{0.4cm}
{\LARGE {\bf Politechnika Warszawska}}\\
\vspace{0.3cm}
\end{center}
\vspace{5cm}
\begin{center}
{\bf \LARGE Wprowadzenie do sztucznej inteligencji \vskip 0.1cm}
\end{center}
\vspace{1cm}
\begin{center}
{\bf \LARGE \@title}
\end{center}
\vspace{2cm}
\begin{center}
{\bf \Large \@author \par}
\end{center}
\vspace*{\stretch{6}}
\begin{center}
\bf{\large{Warszawa, \@date\vskip 0.1cm}}
\end{center}
\end{titlepage}
}
\makeatother

\maketitle

\tableofcontents
% !TEX encoding = cp1250
\chapter{Wst�p}

W niniejszym sprawozdaniu opisane zosta�o rozwi�zanie zadania oraz eksperymenty dotycz�ce zadania nr 6 polegaj�cego na implementacji naiwnego klasyfikatora Bayesa. Algorytm ten zosta� zastosowany do klasyfikacji odmian kosa�c�w na podstawie parametr�w takich jak d�ugo�� dzia�ki kielicha czy p�atk�w. Dane zosta�y pobrane z \url{https://archive.ics.uci.edu/ml/datasets/iris} za pomoc� pakietu \verb!sklearn!.
% !TEX encoding = cp1250


\chapter{�wiczenie}

\section{Eksperymenty}
Hiperparametrami zmienianymi podczas eksperyment�w by�y:
\begin{itemize}
\item wielko�� populacji $\mu$ - w zakresie od 10 do 500
\item liczba iteracji algorytmu - w zakresie od 2 do 10
\item prawdopodobie�stwo krzy�owania $p_c$ - od \num{0,0000001} do \num{0,2}
\item prawdopodobie�stwo mutacji $p_m$ - od \num{0,0000001} do \num{0,4}
\end{itemize}

Konkretnymi zestawami wykorzystanymi podczas eksperyment�w by�y:
\begin{itemize}
\item $H_1$: iteracje = 10, $\mu$ = 10, $p_m$ = \num{0,1}, $p_c$ = \num{0,1}
\item $H_2$: iteracje = 10, $\mu$ = 100, $p_m$ = \num{0,1}, $p_c$ = \num{0,1}
\item $H_3$: iteracje = 3, $\mu$ = 50, $p_m$ = \num{0,01}, $p_c$ = \num{0,01}
\item $H_4$: iteracje = 2, $\mu$ = 50, $p_m$ = \num{0,2}, $p_c$ = \num{0,4}
\item $H_5$: iteracje = 2, $\mu$ = 100, $p_m$ = \num{0,2}, $p_c$ = \num{0,4}
\item $H_6$: iteracje = 5, $\mu$ = 100, $p_m$ = \num{0,001}, $p_c$ = \num{0,001}
\item $H_7$: iteracje = 7, $\mu$ = 70, $p_m$ = \num{0,001}, $p_c$ = \num{0,001}
\item $H_8$: iteracje = 10, $\mu$ = 100, $p_m$ = \num{0,0000001}, $p_c$ = \num{0,0000001}
\item $H_9$: iteracje = 2, $\mu$ = 500, $p_m$ = \num{0,001}, $p_c$ = \num{0,001}
\end{itemize}

\section{Funkcja celu}
Funkcja celu osi�ga swoje maksimum w�wczas, gdy rakieta osi�ga wysoko�� 750, lub, je�li osi�gni�cie dok�adnie takiej wysoko�ci, nast�pna mo�liwa wysoko�� wy�sza ni� 750.
Wysoko�� ko�cowa mniejsza ni� 750 skutkuje automatycznie warto�ci� funkcji celu r�wn� 0.
W pozosta�ych przypadkach warto�� funkcji celu jest zale�na od ilo�ci zabranego paliwa, gdy� w�wczas wynosi $200-x$, gdzie $x$ to liczba zabranych jednostek paliwa.


\section{Wyniki}
Ka�dy zestaw hiperparametr�w pos�u�y� do przeprowadzenia 25 pomiar�w.

\begin{center}
\begin{tabular}{ |c|c|c|c|c|c| } 
 \hline
 &liczba iteracji&$\mu$&$p_m$&$p_c$&�redni wynik\\
 \hline
$H_1$ & 10 & 10 & \num{0,1}&\num{0,1} & \num{101,12}\\ 
 \hline
 $H_2$ & 10 & 100 & \num{0,1}&\num{0,1} & \num{110,28}\\ 
 \hline
 $H_3$ & 3 & 50 & \num{0,01}&\num{0,01} & \num{112,52}\\ 
 \hline
 $H_4$ & 2 & 50 & \num{0,2}&\num{0,4} & \num{112,8}\\ 
 \hline
 $H_5$ & 2 & 100 & \num{0,2}&\num{0,4} & \num{115,8}\\ 
 \hline
 $H_6$ & 5 & 100 & \num{0,001}&\num{0,001} & \num{113,52}\\ 
 \hline
 $H_7$ & 7 & 70 & \num{0,001}&\num{0,001} & \num{110,92}\\ 
 \hline
 $H_8$ & 10 & 100 & \num{0,0000001}&\num{0,0000001} & \num{110,8}\\ 
 \hline
  $H_9$ & 2 & 500 & \num{0,001}&\num{0,001} & \num{119,52}\\ 
  \hline
\end{tabular}
\end{center}

\section{Analiza wynik�w}
Algorytm spisywa� si� najlepiej przy niskiej liczbie generacji. Wyniki s� do�� por�wnywalne, przy czym warto zauwa�y�, �e najwy�sze �rednie zosta�y uzyskane przy najwi�kszych populacjach. Wynika to z faktu, �e w�wczas posiadamy najwi�ksz� pul�, z kt�rej mo�na uzyska� rozwi�zania. Przy wy�szych liczbach iteracji wyst�powa�o zbyt du�o krzy�owa� i mutacji, co skutkowa�o wysok� losowo�ci� otrzymanych wynik�w. Widoczna jest warto�� dodana wzgl�dem zupe�nie losowej populacji, gdy� dla takiej warto�� �rednia wynios�a \num{105,84}.

\section{Wnioski}
Algorytm genetyczny spisuje si� dobrze w takim zadaniu, jednak �atwo wyobrazi� sobie, �e przy zadaniach ,,trudniejszych'', czyli takich, w kt�rych mniejsza liczba kombinacji zapewnia niezerow� warto�� funkcji celu, algorytm ten mo�e by� zupe�nie nieskuteczny i ,,wypluwa�'' same zera.
Wynika to z natury tego algorytmu, kt�ra sprawia, �e �atwo mo�e doj�� do ,,zgubienia'' niezerowych rozwi�za� podczas krzy�owania lub mutacji.
Tak� wad� mo�na jednak �atwo skorygowa�, na przyk�ad poprzez:
\begin{itemize}
\item przechwytywanie ka�dego niezerowego wyniku i por�wnywanie wszystkich uzyskanych na koniec dzia�ania programu
\item wykluczanie najlepszego osobnika w ka�dej iteracji z krzy�owania i mutacji
\end{itemize}
\end{document}

